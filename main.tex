%\title{My two column CV}
%
% tccv (two columns curriculum vitae) is a LaTeX class inspired by
% the template found at latextemplates.com by Alessandro Plasmati.
%
% Create by Nicola Fontana, the original files can be downloaded from:
% http://dev.entidi.com/p/tccv/
%
\documentclass{tccv}
\usepackage[spanish]{babel}
\selectlanguage{spanish}
\usepackage[utf8]{inputenc}

\begin{document}

\part{Juan David López Sánchez}


\personal
    [https://www.facebook.com/jdlopezs]
    {Carera 8 No. 9 - 37, Facatativá.}
    {+57 322 361 0273}
    {jdlopezs12@gmail.com}


\section{Experiencia Laboral}

\vspace{10pt}

\begin{eventlist}

\item{Febrero 2018 - Marzo 2018}
     {Voluntariado}
     {Científico de datos Junior}

Cargo: Aprendiz voluntario en el análisis e interpretación de grandes cantidades de datos para identificar formas de mejorar las organizaciones, a través de machine learning y modelos estadísticos.\hfill 

Con responsabilidades como: \newline Realizar EDA (Exploratory-Descriptive Analysis) de diversos grupos de datos. Desarollar y programar algoritmos de machine learning y diseñar modelos estadísticos para soportar la toma de decisiones. 

\vspace{10pt}
\item{Agosto 2018 -- Presente}
     {Avianca S.A.}
     {Aprendiz HUB Control}
Acompañamiento en el área de coordinacion de operaciones terrestres conocida como HUB Control, realizando análisis y  levantamiento de procesos, Acuerdos de Nivel de Servicio, acompañano procesos de ergonomía y 
\end{eventlist}
\vspace{6pt}
\section{Educación}

\begin{yearlist}
\item{2014 -- Presente}
     {Ingeniería Industrial}
     {Universidad Nacional de Colombia.}

\item{2011 -- 2013}
     {Ingeniería Industrial y Física}
     {Universidad de  los \newline Andes.}


\end{yearlist}
\vfill\null
\columnbreak

\section{Resumen}
\begin{eventlist}
\item{Universidad Nacional de Colombia}
{  }
{  }

Ingeniero industrial con orientación a la optimización, data science y Big Data en general; cuento con aptitud para la programación y buenas bases estadísticas. Mis intereses principales en cuanto a área de trabajo/aprendizaje son: pensamiento sistémico, redes sociales, machine learning, divulgacion científica, inteligencia artificial, inferencia social y data mining. 
\end{eventlist}
\vspace{-6pt}
\section{Habilidades de comunicación}

\begin{factlist}
\item{Español}{Nativo}
\item{Inglés}{Alto}
\item{Alemán}{Básico}
\item{Francés}{Pobre}
\end{factlist}

\section{Habilidades de programación}

\begin{yearlist}

\item{Intermedio}
     {C, Python, Java, VBA}
     {}

\item{Básico}
     {SpSS, Latex, AnyLogic, R}
     {}

\end{yearlist}

\vspace{-6pt} % Otherwise it just falls onto the next page.
\section{Referencias}


\begin{eventlist}

\item{Luis Vicente López Gil. }
     { Ingeinero de intrgracion y arquitectura SOA \newline
     Hospital San Ignacio  \newline
     (+57) 310 254 8277}
     { }

\vspace{-6pt}
\item{ Carlos Eduardo Sánchez Méndez. }
     {Jefe de planeación y producción poliductos. \newline Cenit Transporte. \newline
     (+57) 320 696 7480}
     { }
    
\vspace{-6pt} 
\item{Davi Felipe Medina Mayorga. }
     {Profesor de modelos estocáticos y simulación. \newline Universidad Nacional de Colombia. \newline
     (+57) 310 551 1463}
     { }
\end{eventlist}

\end{document}
